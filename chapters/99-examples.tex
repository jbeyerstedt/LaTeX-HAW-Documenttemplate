% !TEX root = ../main.tex

\chapter{Examples}
Just a collection of random examples using the included packages.

\section{Standard elements and text tools}

\subsection{Bibliography and Citations}
Citations are pretty easy. \cite{AEEE:2016}


\subsection{Maths}
\LaTeX{} is great at typesetting mathematics. Let $X_1, X_2, \ldots, X_n$ be a sequence of independent and identically distributed random variables with $\text{E}[X_i] = \mu$ and $\text{Var}[X_i] = \sigma^2 < \infty$, and let
$$S_n = \frac{X_1 + X_2 + \cdots + X_n}{n}
      = \frac{1}{n}\sum_{i}^{n} X_i$$
denote their mean. Then as $n$ approaches infinity, the random variables $\sqrt{n}(S_n - \mu)$ converge in distribution to a normal $\mathcal{N}(0, \sigma^2)$.



\subsubsection{German Quotes}
For german style quotes you can use the \code{enquote} command.
Here is some \enquote{example in german quotes}.
On the other hand, there can be text in 'single quotes' or ''two single quotes''.


\subsubsection{Inline code}
To create a visual hint, that the following characters are related to code, these can pretty easily be set in a monospaced font.
Of course you can use \code{texttt} for this, but the custom command \code{code} is the better markup.


\subsubsection{Table}
\begin{table}[!h]
    \centering
    \begin{tabular}{l l l}
        %\firsthline
        \textbf{Beschreibung}   & \textbf{ID}           & \textbf{Länge} \\
        \hline
        Magic Number            & \code{magic}          & 4~Byte \\ \hline
        Chip Seriennummer       & \code{chip\_id}       & 16~Byte \\ \hline
    \end{tabular}
    \caption{Example Table}
    \label{tab:example}
\end{table}


\subsubsection{Image}
\begin{figure}[htb]
    \centering
    \includegraphics[width=0.5\textwidth]{logo-haw-2017.png}
    \caption{example figure captions}
    \label{fig:example}
\end{figure}


\subsubsection{Code Listing}
\begin{lstlisting}[caption=Example Listing,label=lst:example]
typedef struct OTA_FW_metadata_t {
    uint32_t magic;
    uint8_t  hw_id[8];
} OTA_FW_metadata_t;
\end{lstlisting}




\section{Additional things}

\subsubsection{TODO notes}
To have a todo at the page margin, simple use the \code{todo} command. \todo{look here}
But there is also a custom command for todo notes, which are placed at full width inside the text.
I called them \code{inlinetodo}.
\inlinetodo{There's enough space for long comments or descriptions of what's missing.}


\subsubsection{Glossary and Abbrevations}
With the \code{glossaries} package you can differentiate between a glossary and a list of acronyms.
Glossary items have an id, a name and can have a quite long description.
Acronyms on the other hand are automatically written as full text the first time they are mentioned in the text.
On all other occurrences, only the abbreviation is used.

\newglossaryentry{CBC}{name=CBC, description={Cipher Block Chaining. Betriebsart des symmetrischen Kryptografie-Algorithmus AES}}
\newglossaryentry{ota_long}{name={Over the Air}, description={Übertragen von Daten über eine drahtlose Schnittstelle. Hier meist im Sinne von drahtlosen Firmware-Updates}}

\newacronym{iot}{IoT}{Internet of Things}
\newacronym[see={[Glossary:]{ota_long}}]{ota}{OTA}{Over the Air\glsadd{ota_long}}

For example, the phrase \gls{iot} is first explained and after that only the abbreviation \gls{iot} is used.
This explanations does not occur for glossary entries.

To have the phrase begin with an upper case letter, use \code{Gls} (\Gls{CBC}).
For a plural, use \code{glspl} (\glspl{CBC})

Of course there is also a command for manually getting the full description \code{acrlong} (\acrlong{iot}) or only the abbreviation \code{acrshort} (\acrshort{ota}).
To have the full introduction of the acronym, type \code{acrfull} (\acrfull{iot}).


\subsection{Enumerations}
With the \code{enumitem} package, it is possible to start an enumeration at one place and continue it later in the document.
And the numbers can be customized:

\begin{enumerate}[label=\textbf{R\arabic*},rightmargin=0.7cm,series=basicreq]
\item \label{req:example1}
Some requirement.
\item \label{req:example2}
Some othe requirement.
\end{enumerate}

Here is a normal enumeration:
\begin{enumerate}
    \item first
    \item and second
\end{enumerate}

Or with less space:
\begin{enumerate}[noitemsep]
    \item first
    \item and second
\end{enumerate}

And we continue the first enumeration:
\begin{enumerate}[basicreq]
\item \label{req:example3}
Tho lines:\\
Are no problem at all (\vgl~\cite{AllegroSoft:2011}).
\item \label{req:example4}
Another example.
\end{enumerate}
